% Created 2023-07-26 Wed 17:11
% Intended LaTeX compiler: pdflatex

% =================================BASE====================================%
\documentclass[10pt]{article}
\usepackage[left=2cm,right=2cm,top=2cm,bottom=2cm]{geometry} % Marges
%\usepackage{libertine}
%\usepackage{libertinust1math}
\usepackage[T1]{fontenc} % Nécessaire avec FrenchBabel
\usepackage[utf8]{inputenc} % Important pour symboles Francophones, é,à,etc

\usepackage{lmodern}
\renewcommand{\familydefault}{cmr} % La meilleure police (CMU Serif Roman) (Je me suis battu).

\usepackage{natbib} % Bibliographie
\bibliographystyle{abbrvnat}



\usepackage{amsmath, amssymb, amsthm} % Symb. math. (Mathmode+Textmode) + Beaux théorèmes.

\usepackage{mathtools,cancel} % Utilisation de boîtes \boxed{} + \cancelto{}{}
\usepackage{graphicx, wrapfig} % Géstion des figures.
\usepackage{hyperref} % Permettre l'utilisation d'hyperliens.
\usepackage{color} % Permettre l'utilisation des couleurs.
\usepackage[dvipsnames]{xcolor} % Couleurs avancées.
\usepackage{titling} % Donne accès à \theauthor, \thetitle, \thedate

% >>> Physique >>>
\usepackage{physics} % Meilleur package pour physicien. 
\usepackage{pxfonts} % Rajoute PLEIN de symboles mathématiques, dont les intégrales doubles et triples
% <<< Physique <<<

\usepackage{lipsum} % For fun
\usepackage{tikz} % Realisation de figures TIKZ.
\usepackage{empheq} % Boite autour de MULTIPLE équations

\usepackage[french]{babel} % Environnements en Français.
% ==============================BASE-(END)=================================%



% ================================SETTINGS=================================%
% Pas d'indentation en début de paragraphe :
\setlength\parindent{0pt} 

% Couleurs de hyperliens :
\definecolor{mypink}{RGB}{147, 0, 255}
\hypersetup{colorlinks, urlcolor=mypink, citecolor=mypink, linkcolor=mypink}

% Numéros d'équations suivent les sections :
\numberwithin{equation}{section} 

% Les « captions » sont en italique et largeur limitée
\usepackage[textfont = it]{caption} 
\captionsetup[wrapfigure]{margin=0.5cm}

% Retirer le l'écriture en gras dans la table des matières
\usepackage{tocloft}
\renewcommand{\cftsecfont}{\normalfont}
\renewcommand{\cftsecpagefont}{\normalfont}

% Change bullet style
\usepackage{pifont}
\usepackage{enumitem}
%\setlist[itemize,1]{label=\ding{224}}
\setlist[itemize,1]{label=\ding{239}}
\renewcommand{\boxtimes}{\blacksquare}
% ================================SETTINGS=================================%



% ==============================NEWCOMMANDS================================%
% Degrés Celsius :
\newcommand{\celsius}{${}^\circ$ C} % \degrée Celsius : Pas mal plus simple qu'utilise le package gensymb qui plante avec tout...

% Vecteurs de base :
\newcommand{\nvf}{\vb{\hat{n}}}
\newcommand{\ivf}{\vb{\hat{i}}}
\newcommand{\jvf}{\vb{\hat{j}}}
\newcommand{\kvf}{\vb{\hat{k}}}

\newcommand{\uu}{\vb*{u}}
\newcommand{\vv}{\vb*{v}}

% Boîte vide pour ajuster les underbrace
\newcommand{\bigno}{\vphantom{\qty(\frac{d}{q})}}
\newcommand{\pt}{\hspace{1pt}}

% Physics empty spaces 
\newcommand{\typical}{\vphantom{A}}
\newcommand{\tall}{\vphantom{A^{x^x}_p}}
\newcommand{\grande}{\vphantom{\frac{1}{xx}}}
\newcommand{\venti}{\vphantom{\sum_x^x}}

% Moyenne numérique entre deux points de grilles. 
\newcommand{\xmean}[1]{\overline{#1}^x}
\newcommand{\ymean}[1]{\overline{#1}^y}
\newcommand{\zmean}[1]{\overline{#1}^z}
\newcommand{\xymean}[1]{\overline{#1}^{xy}}

% Tilde over psi
\newcommand{\tpsi}{\tilde{\psi}}

% Nota Bene env :
\newcommand{\nb}{\textbf{N.B.}\hspace{4pt}}
   
% ==============================NEWCOMMANDS================================%



% ==============================PAGE-TITRE=================================%
% Titlepage 
\newcommand{\mytitlepage}{
\begin{titlepage}
\begin{center}
{\Large Contrat Été 2023 \par}
\vspace{2cm}
{\Large \MakeUppercase{\thetitle} \par}
\vspace{2cm}
RÉALISÉ DANS LE CADRE\\ D'UN PROJET POUR \par
\vspace{2cm}
{\Large ISMER--UQAR \par}
\vspace{2cm}
{\thedate}
\end{center}
\vfill
Rédaction \\
{\theauthor}\\
\url{charles-edouard.lizotte@uqar.ca}\\
ISMER-UQAR
\end{titlepage}
}
% ==============================PAGE-TITRE=================================%



% =================================ENTÊTE==================================%
\usepackage{fancyhdr}
\pagestyle{fancy}
\setlength{\headheight}{13pt}
\renewcommand{\headrulewidth}{1.3pt} % Ligne horizontale en haut

\fancyhead[R]{\textit{\thetitle}}
\fancyhead[L]{\ \thepage}
\fancyfoot[R]{\textit{\theauthor}}
\fancyfoot[L]{}
\fancyfoot[C]{} 
% =================================ENTÊTE==================================%
\author{Charles-Édouard Lizotte}
\date{28/07/2023}
\title{Carnet de bord, Université McGill}
\hypersetup{
 pdfauthor={Charles-Édouard Lizotte},
 pdftitle={Carnet de bord, Université McGill},
 pdfkeywords={},
 pdfsubject={},
 pdfcreator={Emacs 27.1 (Org mode 9.6.7)}, 
 pdflang={French}}
\begin{document}

\mytitlepage
\tableofcontents\newpage



\section{Mettre à jour portable pour travail à distance  \textit{<2023-07-24 Mon>}}
\label{sec:org1e32711}

De manière générale, j'utilise l'ordinateur de bureau fourni par l'université McGill pour travailler.
Tout ça me permet d'avoir un meilleurs service sur l'installation des module (\emph{packages}).
La proximité de Ambrish, soit le James Caveen de McGill, m'aide beaucoup à comprendre comment tout ça fonctionne.
Cependant, je n'ai pas pris le temps de mettre cet ordinateur portable à jour.
Aujourd'hui est donc une bonne occasion de le faire.
\begin{itemize}
\item[{$\boxtimes$}] Cloner le \emph{.emacs.d} ;
\item[{$\boxtimes$}] Mise à jour de tous les modules MELPA pour Emacs ;
\item[{$\boxtimes$}] Installer les packages \LaTeX{}.
\item[{$\boxtimes$}] Cloner la dossier Documentation et tous les rapports ;
\item[{$\boxtimes$}] Cloner la branche \emph{walls} du modèle \emph{shallow water} ;
\end{itemize}


\section{Retour sur les conditions frontières \textit{<2023-07-26 Wed>}}
\label{sec:orga3b15ef}

\begin{wrapfigure}[20]{r}{0.45\textwidth}
\vspace{-\baselineskip}
\begin{center}
\begin{tikzpicture}
%
\foreach \i in {0,1,2,3}
{\foreach \j in {-1,0,1,2,3}
{\draw [thick, red!40] (\i,\j+1) -- (\i,\j) ;
 \draw [thick,blue!40] (\j,\i) -- (\j+1,\i) ;}}
%
\foreach \i in {0,1,2,3}
{\foreach \j in {-1,0,1,2,3}
{\draw [-latex,thin,red!40 ] (\i,0.5+\j) -- (\i+0.15,0.5+\j);
 \draw [-latex,thin,blue!40] (0.5+\j,\i) -- (0.5+\j,\i+0.15);}}
% Domaine (Rectangle autour)
\draw [dotted] (0,0) rectangle (3,3);
% Cercles (Centres) :
\foreach \i in {0,1,2,3,4}
{\foreach \j in {0,1,2,3,4}
{\fill[fill=black ] (\i-0.5,\j-0.5) circle (0.9pt);}}
% Rectangles (Noeuds) :
\foreach \i in {0,1,2,3}
\foreach \j in {0,1,2,3}
{{\filldraw [black!85] (\i-0.03,\j-0.03) rectangle (\i+0.03,\j+0.03);}}
% Rulers 
\draw[> = latex, arrows = {|<->|}, thin,red ] (0,4.5) -- (3,4.5);
\draw (1.5,4.5) node [above,red] {nx};
\draw[> = latex, arrows = {|<->|}, thin,blue] (-1.5,0) -- (-1.5,3);
\draw (-1.5,1.5) node [above,blue,rotate=90] {ny};
\draw[> = latex, arrows = {|<->|}, thin,black!50] (4.5,-1) -- (4.5,4);
\draw (4.5,1.5) node [below,black!50,rotate=90] {$[0:\text{ny}+1]$};
\draw[> = latex, arrows = {|<->|}, thin,black ] (-0.5,-1.5) -- (3.5,-1.5);
\draw (1.5,-1.5) node [below,black] {$[0,\text{nx}]$};
%
pp\draw (0.5,-0.15) node [] {\color{blue}\tiny$v\pt(1,1)$};
\draw (-0.15,0.5) node [rotate=90] {\color{red}\tiny$u\pt(1,1)$};
% Ghost points (mailles):
\draw [black!20] (-1,-1) rectangle (4,4);
\foreach \i in {0,1,2,3,4}
{
\draw [-latex,thin,black!20] (-1,-0.5+\i) -- (-0.85,-0.5+\i);
\draw [-latex,thin,black!20] (4.0,-0.5+\i) -- (4.15,-0.5+\i);
\draw [-latex,thin,black!20] (-0.5+\i,-1) -- (-0.5+\i,-0.85);
\draw [-latex,thin,black!20] (-0.5+\i,4) -- (-0.5+\i,4.15);
}
% Ghost points (noeuds) :
\foreach \i in {-1,0,1,2,3,4}
{
\filldraw [black!20] (-1-0.03,\i-0.03) rectangle (-1+0.03,\i+0.03);
\filldraw [black!20] (4-0.03,\i-0.03) rectangle (4+0.03,\i+0.03);
\filldraw [black!20] (\i-0.03,-1-0.03) rectangle (\i+0.03,-1+0.03);
\filldraw [black!20] (\i-0.03,4-0.03) rectangle (\i+0.03,4+0.03);
};
% Ghost text : 
\draw (-0.5,-1.15) node [gray] {\tiny$v\pt(0,0)$};
\draw (-1.15,-0.5) node [gray,rotate=90] {\tiny$u\pt(0,0)$};
\end{tikzpicture}
\end{center}
\caption{\label{orgb26196a}Exemple de grille avec frontières fixes (nx\pt=\pt ny\pt=\pt4). Pointillé central définit les frontière «\pt réelles\pt» du modèle tandis que tous les points aux alentours sont des points fantômes.}
\end{wrapfigure}

Au \href{rapport-2023-07-07.org}{dernier rapport}, nous avons posé les bases d'un modèle emboîté par de frontières de conditions \emph{no normal flow} et \emph{free slip}.
Autrement dit, aucun courant ne traverse les frontières et la contrainte de cisaillement normales à ces dernières est nulle. \bigskip

Tout juste avant les vacances \footnote{Rencontre effectuée le 14 juillet 2023}, David et Louis-Philippe avaient fortement insisté sur la nécessité de garder tous points de grille fantôme provenant de l'ancien modèle.
Principalement, le but de l'exercice est de voir si des erreurs se glissent entre les lignes de notre code. \bigskip

Un bref résumé des sous-routines à modifier en terme de mailles : 
\begin{itemize}
\item[{$\square$}] \emph{main.f90}
\item[{$\boxtimes$}] \emph{rhs.f90}
\item[{$\boxtimes$}] \emph{p correction.f90}
\item[{$\boxtimes$}] \emph{diags.f90}
\item[{$\boxtimes$}] \emph{init mudpack.f90}
\end{itemize}


\subsection{Tailles des quantités}
\label{sec:org90ae6c0}
Principalement, on peut diviser nos quantités physiques en trois catégories, soient les \textbf{mailles} (\(\rightarrow\)), les \textbf{noeuds} (\(\blacksquare\)) et les \textbf{centres} (\(\bullet\)).
Chacune des quantités physique devra avoir une taille préférentielle.
Bien qu'on me l'aille déconseillé, je tiens à quand même « tenir mon bout » pour qu'on change les tailles de domaine un peu partout.
La raison me motivant est extrêmement simple : nous verrons bien plus facilement les erreurs fatales si ce sont des erreurs de code plutôt que des erreurs mathématiques.
On essaie tout ça aujourd'hui, si ça marche pas, on passe définitivement à autre chose.

\subsubsection{Les mailles (\(\rightarrow\))}
\label{sec:org777fa0a}
Les mailles font référence aux vitesses et au \emph{RHS} de ces dernières. 
Dans la figure \ref{orgb26196a}, ces quantités sont illustrées à l'aide des couleurs bleu et rouge.
\begin{verbatim}
real :: u(0:nx+1,0:ny), v(0:nx,0:ny+1)
\end{verbatim}
Ces quantités ont des tailles non-homogènes, car elles n'ont pas les mêmes orientations.

\subsubsection{Les centres (\(\bullet\))}
\label{sec:org0d6ed4a}
Les centre font référence aux quantités physiques qui se retrouvent au milieu des carrés dans une grille Arakawa-C.
On fait donc référence à la variation de l'interface \(\eta\), à la divergence, etc.
\begin{verbatim}
real :: eta(0:nx,0:ny)
real :: div(0:nx,0:ny)
\end{verbatim}


\subsubsection{Les noeuds (\(\blacksquare\))}
\label{sec:orgedf5251}
Les noeuds font déférence à quantités physiques aux jonctions entre les courants, donc les coins de nos carrés.
Ces quantités sont généralement reliées à un rotationnel, tel que la fonction de courant \(\psi\), la vorticité \(\zeta\) et la fréquence de Coriolis \(f\).
\begin{verbatim}
real :: psi(0:nx+1,0:ny+1)
real :: zeta(0:nx+1,0:ny+1)
\end{verbatim}

\subsection{Mention spéciale pour les boucles (do loop)}
\label{sec:orge64d54a}

Puisque les conditions frontières \emph{no normal flow} et \emph{free slip} contraignent le système à avoir des courants nuls ou répétitifs, on se permet d'itérer entre \(i=2,\pt nx-1\).
Par la suite, on applique ces mêmes conditions frontière pour pallier les points qui n'ont pas été itérés.
D'un côté, ça nous sauve du temps de calcul et de l'autre, ça nous permet de seulement itérer sur ce qu'on a besoin d'itérer.\bigskip

Pour n'en nommer que quelques unes,
\begin{itemize}
\item Les \textbf{mailles} (\emph{edges}) n'ont besoin que d'être mises à jour qu'entre i,\pt j = 1 à nx,\pt ny-1.
C'est nécessaire si l'on veut veut faire nos applications sur tous les courants.
\item Les \textbf{noeuds} (/nodes) sont mis à jours entre i,\pt j = 2 à nx,\pt ny-1, car les fonctions de courants (\(\psi\)) et les vorticitées (\(\zeta\)) sont nulles aux frontières.
\item Les \textbf{centres} sont mis à jour de 1 à nx,\pt ny-1, car le reste est constitué de points fantômes.
\end{itemize}

\subsection{Expansion en série de Taylor pour le Laplacien}
\label{sec:orgbb97765}
Précédemment, nous avions développé l'expression de la dérivée seconde pour les murs.
Nous allons réaliser un petit rappel, car ça «\pt fait un boutte\pt» que je l'ai réalisé.
Sans oublier que j'ai besoin d'une référence accessible juste ici.\bigskip

On réalise deux expansions en série de Taylor depuis le mur pour les premiers et seconds points.
Ainsi
\begin{align}
   &&\boxed{\text{A}} && &u(2) = \cancelto{0}{u(1)} + \Delta x \cdot u'(1) + \qty(\frac{\Delta x^2}{2}) \ u''(1) && &&\\
   &&\boxed{\text{B}} && &u(3) = \cancelto{0}{u(1)} + 2\Delta x \cdot u'(1) + \qty(\frac{4\pt \Delta x^2}{2}) \ u''(1) && &&
\end{align}
Par la suite, on soustrait les équations de sorte à éliminer la dérivée première du courant, soit \(B - 2A\),
\begin{equation}
   u(3) - u(2) = 2\pt \Delta x^2 u''(1) - \Delta x^2 u''(1).
\end{equation}
Au final,
\begin{equation}
   \boxed{\hspace{0.3cm} u''(1) = \frac{u(3)-u(2)}{\Delta x^2}.\hspace{0.3cm}}
\end{equation}
Très simple.

\section{Gros ménage du modèle numérique \textit{<2023-07-25 Tue>}}
\label{sec:orgde657ad}
Comme le modèle fonctionnera uniquement avec MUDPACK, on doit purger tout ce qui est relié aux transformées de Fourier.
Principalement, la plupart des quantités et sous-routines qui y sont reliés servent à faire des diagnostiques, donc je crois qu'on peut les retirer sans problème.
Sans oublier que toutes ces sous-routines et quantités existent séparément sur mon \href{https://github.com/charli1213/Modele-shallow-water-multicouches}{GitHub personnel} dans la branche main \footnote{Xavier m'a mentionné qu'il aimerait faire du « shallow water », donc permettons nous de conserver quelques traces du modèle en FFT. En ce moment, la version FFT est toujours vivante et fonctionnelle sur le git, mais elle n'est pas assez propre à mon gout.}.
\end{document}