% Created 2024-06-16 Sun 00:14
% Intended LaTeX compiler: pdflatex
ocumentclass[10pt]{article}% =================================BASE====================================%
\documentclass[10pt]{report}
\usepackage[left=2cm,right=2cm,top=2cm,bottom=2cm]{geometry} % Marges
\usepackage[T1]{fontenc} % Nécessaire avec FrenchBabel
\usepackage[utf8]{inputenc} % Important pour symboles Francophones, é,à,etc
\usepackage{csquotes} % Recommandé par PDFLatex lors de la compilation. 


% Calligraphie
%\usepackage{lmodern} % Ça, ça set latin modern
%\usepackage{mathrsfs} %Permet la command \mathscr (Lettres attachées genre) \mathscr(B)

% Calligraphie
%\usepackage{pxfonts} % Met le texte ET les maths en Palatino + donne accès à des symboles math
\usepackage{palatino} % Cette commande met seulement le texte en police palatino
\usepackage{lmodern} % Pour les maths?
% Use lmodern for sans-serif
\usepackage{mathrsfs} % Permet la command \mathscr (Lettres attachées genre) \mathscr(B)





% Bibliographie
%\usepackage[backend=bibtex,style=phys,sorting=ynt]{biblatex}
\usepackage[backend=biber,sorting=ynt,style=ieee]{biblatex}




\usepackage{amsmath, amssymb, amsthm} % Symb. math. (Mathmode+Textmode) + Beaux théorèmes.
\usepackage{mathtools,cancel,xfrac} % Utilisation de boîtes \boxed{} + \cancelto{}{}, xfrac
\usepackage{graphicx, wrapfig} % Géstion des figures.
\usepackage{hyperref} % Permettre l'utilisation d'hyperliens.
\usepackage{color} % Permettre l'utilisation des couleurs.
\usepackage{colortbl} % Color tables
\usepackage[dvipsnames]{xcolor} % Couleurs avancées.
\usepackage{titling} % Donne accès à \theauthor, \thetitle, \thedate

% Physique
\usepackage{physics} % Meilleur package pour physicien. 


% Style
\usepackage{lipsum} % For fun
\usepackage{tikz} % Realisation de figures TIKZ.
\usepackage{empheq} % Boite autour de MULTIPLE équations
\usepackage{bbding}

% Français
\usepackage[french]{babel} % Environnements en Français.
% ==============================BASE-(END)=================================%



% ================================SETTINGS=================================%
% Pas d'indentation en début de paragraphe :
\setlength\parindent{0pt}
\setlength{\parskip}{0.15cm}

% Tableaux/tabular
% Espace vertical dans les tabular/tableaux
\renewcommand{\arraystretch}{1.2}
% Couleur des tableaux/tabular
\rowcolors{2}{violet!5}{}

% Couleurs de hyperliens :
\definecolor{mypink}{RGB}{147, 0, 255}
\hypersetup{colorlinks, 
             filecolor=mypink,
             urlcolor=mypink, 
             citecolor=mypink, 
             linkcolor=mypink, 
             anchorcolor=mypink}


\usepackage{titling} % Donne accès à \theauthor, \thetitle, \thedate

% Physique
\usepackage{physics} % Meilleur package pour physicien. 


% Style
\usepackage{lipsum} % For fun
\usepackage{tikz} % Realisation de figures TIKZ.
\usepackage{empheq} % Boite autour de MULTIPLE équations

% Français
\usepackage[french]{babel} % Environnements en Français.
% ==============================BASE-(END)=================================%





% ================================SETTINGS=================================%
% Pas d'indentation en début de paragraphe :
\setlength\parindent{0pt}
\setlength{\parskip}{0.15cm}

% Tableaux/tabular
% Espace vertical dans les tabular/tableaux
\renewcommand{\arraystretch}{1.2}
% Couleur des tableaux/tabular
\rowcolors{2}{violet!5}{}

% Couleurs de hyperliens :
\definecolor{mypink}{RGB}{147, 0, 255}
\hypersetup{colorlinks, 
             filecolor=mypink,
             urlcolor=mypink, 
             citecolor=mypink, 
             linkcolor=mypink, 
             anchorcolor=mypink}


% Numéros d'équations suivent les sections :
\numberwithin{equation}{section} 

% Les « captions » sont en italique et largeur limitée
\usepackage[textfont = it]{caption} 
\captionsetup[wrapfigure]{margin=0.5cm}

% Retirer l'écriture en gras dans la table des matières
\usepackage{tocloft}
\renewcommand{\cftsecfont}{\normalfont}
\renewcommand{\cftsecpagefont}{\normalfont}

% Change bullet style
\usepackage{pifont}
\usepackage{enumitem}
%\setlist[itemize,1]{label=\ding{224}}
\setlist[itemize,1]{label=\ding{239}}
\renewcommand{\boxtimes}{\blacksquare}
% ================================SETTINGS=================================%



% ==============================NEWCOMMANDS================================%

% Vecteurs de base :
\newcommand{\nvf}{\vb{\hat{n}}}
\newcommand{\ivf}{\vb{\hat{i}}}
\newcommand{\jvf}{\vb{\hat{j}}}
\newcommand{\kvf}{\vb{\hat{k}}}
\newcommand{\uu}{\vb{u}}
\newcommand{\vv}{\vb{v}}
\newcommand{\ust}{\vb{u}_{\ast}}

% Physics empty spaces 
\newcommand{\typical}{\vphantom{A}}
\newcommand{\tall}{\vphantom{A^{x^x}_p}}
\newcommand{\grande}{\vphantom{\frac{1}{xx}}}
\newcommand{\venti}{\vphantom{\sum_x^x}}
\newcommand{\pt}{\hspace{1pt}} % One horizontal pt space

% Moyenne numérique entre deux points de grilles. 
\newcommand{\xmean}[1]{\overline{#1}^x}
\newcommand{\ymean}[1]{\overline{#1}^y}
\newcommand{\zmean}[1]{\overline{#1}^z}
\newcommand{\xymean}[1]{\overline{#1}^{xy}}

% Tilde over psi
\newcommand{\tpsi}{\tilde{\psi}}
\newcommand{\tphi}{\tilde{\phi}}

% Nota Bene env : (\ding{89})
%\newcommand{\nb}{$\boxed{\text{\footnotesize\EightStarConvex}\pt \mathfrak{N. B.}}$\hspace{4pt}}
\newcommand{\nb}{\underline{{\footnotesize\EightStarConvex}\pt $\mathfrak{N.B.}$\vphantom{p}}\hspace{3pt}}


% Define the nota bene environment
\usepackage{tcolorbox}
\newtcolorbox{notabene}{
     colback=blue!5,
     colframe=black,
     boxrule=0.5pt,
     arc=2pt,
     left=5pt,
     right=5pt,
     top=5pt,
     bottom=5pt,
}


\newcommand{\cmark}{\ding{52}}
\newcommand{\xmark}{\ding{55}}
% ==============================NEWCOMMANDS================================%



% ==============================PAGE-TITRE=================================%
% Titlepage 
\newcommand{\mytitlepage}{
\begin{titlepage}
\begin{center}
{\Huge Contrat Été 2023 \par}
\vspace{2cm}
{\Huge \MakeUppercase{\thetitle} \par}
\vspace{2cm}
RÉALISÉ DANS LE CADRE\\ D'UN PROJET POUR \par
\vspace{2cm}
{\Huge ISMER--UQAR \par}
\vspace{2cm}
{\thedate}
\end{center}
\vfill
Rédaction \\
{\theauthor}\\
\url{charles-edouard.lizotte@uqar.ca}\\
ISMER-UQAR\\
Police d'écriture : \textbf{CMU Serif Roman}
\end{titlepage}
}
% ==============================PAGE-TITRE=================================%



% =================================ENTÊTE==================================%
\usepackage{fancyhdr}
\pagestyle{fancy}
\setlength{\headheight}{13pt}
\renewcommand{\headrulewidth}{0.025pt} % Ligne horizontale en haut

\fancyhead[R]{\textit{\thetitle}}
\fancyhead[L]{\ \thepage}
\fancyfoot[R]{\textit{\theauthor}}
\fancyfoot[L]{}
\fancyfoot[C]{} 
% =================================ENTÊTE==================================%
\author{Charles-Édouard Lizotte}
\date{10/03/2023}
\title{Rapport hebdomadaire -- McGill\\\medskip
\large Semaine du 3 mars 2023}
\hypersetup{
 pdfauthor={Charles-Édouard Lizotte},
 pdftitle={Rapport hebdomadaire -- McGill},
 pdfkeywords={},
 pdfsubject={},
 pdfcreator={Emacs 27.1 (Org mode 9.6.7)}, 
 pdflang={French}}
\begin{document}

\mytitlepage
\tableofcontents\newpage


\section{{\bfseries\sffamily DONE} Organisation du travail [3/3]}
\label{sec:orgb8265ad}
Lors de ma maîtrise -- de 2019 à 2022 --, j'ai constaté que mon organisation du travail était extrêmement maladroite.
Entre-autres, j'avais une grande difficulté à recenser les tâches accomplies ou incomplètes. 
C'est pourquoi j'ai créé ce \emph{template} en \textbf{org-mode}, un language d'écriture \textbf{Emacs} exportable en \LaTeX{}.
Ces petits résumés agissent principalement comme un cahier de laboratoire. 
Ils me permettront donc d'écrire ce que j'ai retenu de nos rencontres, les idées à développer et les tâches à accomplir.\\[0pt]

Tâches : 
\begin{itemize}
\item[{$\boxtimes$}] Apprendre le langage \textbf{org-mode} en construisant quelques tutoriels.
\item[{$\boxtimes$}] Commencer à utiliser le \textbf{org-agenda}.
\item[{$\boxtimes$}] Créer ce \emph{template} en \LaTeX{} et commencer les rapports hebdomadaires.
\end{itemize}

\section{Retour sur le modèle \emph{shallow water} [3/3]}
\label{sec:orgca6d3f7}
\subsection{{\bfseries\sffamily DONE} Télécharger le modèle \emph{shallow water} et créer un \emph{git} [2/2]}
\label{sec:orgd55dc5b}
Avant tout, il faut se reconnecter sur la grappe de calcul \textbf{mingan} pour aller chercher le modèle \emph{shallow-water}. 
Pour des raisons obscures, mon compte mingan ne fonctionnait plus.
J'ai du contacter le gestionnaire du soutient technique de l'UQAR, Dany Lemay, pour créer un nouvel identifiant. 
Sur Pélerin, les commandes de connexion sont maintenant : 
\begin{verbatim}
   >>> ssh -Y lizoch01@calculs.uqar.ca
   >>> <mot-de-passe>
\end{verbatim}

Une fois connecté sur le noeud \emph{Calculs}, on peut entrer sur \emph{mingan} à l'aide des commandes
\begin{verbatim}
   >>> ssh -Y celiz2@mingan.uqar.ca
   >>> <mot-de-passe>
\end{verbatim}
À noter \emph{git} ne fonctionne plus sur \emph{mingan}, donc il a fallut tout transférer sur le noeud \emph{calculs} avant de mettre quelque chose sur le \emph{git}.\\[0pt]

Étapes : 
\begin{itemize}
\item[{$\boxtimes$}] Créer un git et mettre le modèle \emph{shallow-water} sur \emph{Github}.
\item[{$\boxtimes$}] Déplacement des sous-routines de couplage et d'analyse.
\begin{itemize}
\item[{$\boxtimes$}] Déplacer toutes les sous-routines d'analyse Python de ma maîtrise sur \emph{Github} (Y'en a beaucoup)
\item[{$\boxtimes$}] Télécharger les routines modifiées de Wavewatch III sur le \emph{Github}.
\end{itemize}
\end{itemize}

\subsection{{\bfseries\sffamily DONE} Ménage \emph{mingan} et transfert vers \emph{pelerin} [2/2]}
\label{sec:org38c3a25}
Comme la grappe de calculs \emph{mingan} en est à ses derniers heures de vie, je vais devoir transférer tous mes codes vers \emph{pelerin}.
Dany Lemay propose avant tout de créer un \emph{git} pour que ça aille plus rapidement.
À mentionner que ça fait un bout que j'ai laissé ça tarder, j'aurais peut-être du faire ça vers la fin de ma maîtrise..\\[0pt]

Étapes : 
\begin{itemize}
\item[{$\boxtimes$}] Vider le \emph{home/celiz2}
\begin{itemize}
\item[{$\boxtimes$}] Transférer le modèle \emph{shallow-water} du stage 2016 sur \emph{Github}.
\item[{$\boxtimes$}] Faire le ménage et supprimer l'ancienne version de Wavewatch III (WW3-Caroline).
\item[{$\boxtimes$}] Ramener les routines Python de génération \emph{mapsta} et de conditions initiales (BoxPy) sur Pelerin.
\end{itemize}
\item[{$\boxtimes$}] Vider le \emph{share/work/celiz2} :
\begin{itemize}
\item[{$\boxtimes$}] Ramener les routines d'analyse Python du modèle couplé sur \emph{Github}.
\item[{$\boxtimes$}] Filtrer les expériences WW3 à garder et transférer sur le noeud \emph{calculs}.
\end{itemize}
\end{itemize}

\subsection{{\bfseries\sffamily DONE} Rencontre avec David Straub}
\label{sec:orgf0a3eea}
La tâche consiste principalement à modifier le modèle \emph{shallow-water} que David et Yanxu Chen avaient codés, soit celui que j'ai utilisé au cours de ma maîtrise.
Le modèle fonctionne uniquement à 2 couches, car il n'a pas été codé pour en avoir plus, pour l'instant.
Considérant que les \emph{near-inertial waves} s'accumulaient dans la première couche, David et Louis-Philippe ont émis l'hypothèse que l'ajout de plusieurs couches pourrait \emph{damper} la force des ondes quasi-inertielles. 
L'énergie serait graduellement transmise depuis le haut vers le bas en perdant de l'intensité à chaque \emph{étage} sous l'effet de la viscosité.\\[0pt]

\begin{enumerate}
\item Le modèle prévois avoir \emph{n} couches, mais comme la surface est fixe \(\eta_1(i,j)=0\), on peut affirmer qu'on utilise tout simplement pas \(\eta_1\) ou plutôt que \(\eta_1 = 0\) (L390 : \emph{main.f90}). 
En gros, on a la variation de l'épaisseur de la couche partout \(\Delta h(i,j)\), mais il faut retrouver les \(\eta_i\) d'une manière ou d'une autre.
On a \emph{n-1} variations de l'interface, donc il faut trouver un moyen d'imbriquer tout ça dans une \emph{n loop} qui part du bas. 
David a proposé de faire une \textbf{do loop} qui part du bas avec un vecteru temporaire qui s'additionne à chaque étape (voir courriel).

\item Va falloir renommer les \emph{RHS-eta} parce que tous ces \emph{RHS} décrivent plutôt le comportement de \emph{RHS-d}, soit des épaisseurs \(h_k(i,j)\).

\item On va se débarrasser de la couche d'Ekman dans le modèle, car c'est vraiment peu pertinent si on a plusieurs couches.
\end{enumerate}
\end{document}