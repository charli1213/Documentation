% Created 2023-11-16 Thu 16:17
% Intended LaTeX compiler: pdflatex

% =================================BASE====================================%
\documentclass[10pt]{article}
\usepackage[left=2cm,right=2cm,top=2cm,bottom=2cm]{geometry} % Marges
\usepackage[T1]{fontenc} % Nécessaire avec FrenchBabel
\usepackage[utf8]{inputenc} % Important pour symboles Francophones, é,à,etc


% Calligraphie
\usepackage{lmodern}
\renewcommand{\familydefault}{cmr} % La meilleure police (CMU Serif Roman) (Je me suis battu).
%\usepackage{unicode-math} % À réessayer...
%\setmathfont{Latin Modern Math} % À réessayer...
\usepackage{mathrsfs} %Permet la command \mathscr (Lettres attachées genre) mathscr(B)


% Bibliographie
\usepackage[round, sort]{natbib} % Bibliographie
\bibliographystyle{abbrvnat-fr}


\usepackage{amsmath, amssymb, amsthm} % Symb. math. (Mathmode+Textmode) + Beaux théorèmes.
\usepackage{mathtools,cancel,xfrac} % Utilisation de boîtes \boxed{} + \cancelto{}{}
\usepackage{graphicx, wrapfig} % Géstion des figures.
\usepackage{hyperref} % Permettre l'utilisation d'hyperliens.
\usepackage{color} % Permettre l'utilisation des couleurs.
\usepackage{colortbl} % Color tables
\usepackage[dvipsnames]{xcolor} % Couleurs avancées.
\usepackage{titling} % Donne accès à \theauthor, \thetitle, \thedate

% Physique
\usepackage{physics} % Meilleur package pour physicien. 
\usepackage{pxfonts} % Rajoute PLEIN de symboles mathématiques, dont les intégrales doubles et triples

% Style
\usepackage{lipsum} % For fun
\usepackage{tikz} % Realisation de figures TIKZ.
\usepackage{empheq} % Boite autour de MULTIPLE équations

% Français
\usepackage[french]{babel} % Environnements en Français.
% ==============================BASE-(END)=================================%



% ================================SETTINGS=================================%
% Pas d'indentation en début de paragraphe :
\setlength\parindent{0pt}
\setlength{\parskip}{0.15cm}

% Tableaux/tabular
% Espace vertical dans les tabular/tableaux
\renewcommand{\arraystretch}{1.2}
% Couleur des tableaux/tabular
\rowcolors{2}{violet!5}{}

% Couleurs de hyperliens :
\definecolor{mypink}{RGB}{147, 0, 255}
\hypersetup{colorlinks, 
             filecolor=mypink,
             urlcolor=mypink, 
             citecolor=mypink, 
             linkcolor=mypink, 
             anchorcolor=mypink}



% Numéros d'équations suivent les sections :
\numberwithin{equation}{section} 

% Les « captions » sont en italique et largeur limitée
\usepackage[textfont = it]{caption} 
\captionsetup[wrapfigure]{margin=0.5cm}

% Retirer l'écriture en gras dans la table des matières
\usepackage{tocloft}
\renewcommand{\cftsecfont}{\normalfont}
\renewcommand{\cftsecpagefont}{\normalfont}

% Change bullet style
\usepackage{pifont}
\usepackage{enumitem}
%\setlist[itemize,1]{label=\ding{224}}
\setlist[itemize,1]{label=\ding{239}}
\renewcommand{\boxtimes}{\blacksquare}
% ================================SETTINGS=================================%



% ==============================NEWCOMMANDS================================%

% Vecteurs de base :
\newcommand{\nvf}{\vb{\hat{n}}}
\newcommand{\ivf}{\vb{\hat{i}}}
\newcommand{\jvf}{\vb{\hat{j}}}
\newcommand{\kvf}{\vb{\hat{k}}}
\newcommand{\uu}{\vb{u}}
\newcommand{\vv}{\vb{v}}
\newcommand{\ust}{\vb{u}_{\ast}}

% Physics empty spaces 
\newcommand{\typical}{\vphantom{A}}
\newcommand{\tall}{\vphantom{A^{x^x}_p}}
\newcommand{\grande}{\vphantom{\frac{1}{xx}}}
\newcommand{\venti}{\vphantom{\sum_x^x}}
\newcommand{\pt}{\hspace{1pt}} % One horizontal pt space

% Moyenne numérique entre deux points de grilles. 
\newcommand{\xmean}[1]{\overline{#1}^x}
\newcommand{\ymean}[1]{\overline{#1}^y}
\newcommand{\zmean}[1]{\overline{#1}^z}
\newcommand{\xymean}[1]{\overline{#1}^{xy}}

% Tilde over psi
\newcommand{\tpsi}{\tilde{\psi}}
\newcommand{\tphi}{\tilde{\phi}}

% Nota Bene env : (\ding{89})
\newcommand{\nb}{\raisebox{0.8pt}{\scriptsize\textleaf}\ $\mathscr{N. B.}$\hspace{4pt}}
\newcommand{\cmark}{\ding{52}}
\newcommand{\xmark}{\ding{55}}
% ==============================NEWCOMMANDS================================%



% ==============================PAGE-TITRE=================================%
% Titlepage 
\newcommand{\mytitlepage}{
\begin{titlepage}
\begin{center}
{\Large Contrat Été 2023 \par}
\vspace{2cm}
{\Large \MakeUppercase{\thetitle} \par}
\vspace{2cm}
RÉALISÉ DANS LE CADRE\\ D'UN PROJET POUR \par
\vspace{2cm}
{\Large ISMER--UQAR \par}
\vspace{2cm}
{\thedate}
\end{center}
\vfill
Rédaction \\
{\theauthor}\\
\url{charles-edouard.lizotte@uqar.ca}\\
ISMER-UQAR\\
Police d'écriture : \textbf{CMU Serif Roman}
\end{titlepage}
}
% ==============================PAGE-TITRE=================================%



% =================================ENTÊTE==================================%
\usepackage{fancyhdr}
\pagestyle{fancy}
\setlength{\headheight}{13pt}
\renewcommand{\headrulewidth}{0.025pt} % Ligne horizontale en haut

\fancyhead[R]{\textit{\thetitle}}
\fancyhead[L]{\ \thepage}
\fancyfoot[R]{\textit{\theauthor}}
\fancyfoot[L]{}
\fancyfoot[C]{} 
% =================================ENTÊTE==================================%
\author{Charles-Édouard Lizotte}
\date{11/11/2023}
\title{Carnet de bord, Université McGill}
\hypersetup{
 pdfauthor={Charles-Édouard Lizotte},
 pdftitle={Carnet de bord, Université McGill},
 pdfkeywords={},
 pdfsubject={},
 pdfcreator={Emacs 27.1 (Org mode 9.6.7)}, 
 pdflang={French}}
\begin{document}

\mytitlepage
\tableofcontents\newpage

\section{Gestion de l'épaisseur et transfert de momentum du vent -- \textit{<2023-11-14 Tue>}}
\label{sec:org8b0896c}

Traditionnellement, la contrainte de cisaillement ou les stress associé au vent est exprimé en différence finit comme suit,
\begin{align}
   &&\mathrm{RHS}\pt\tau^x = \qty(\frac{1}{\rho_O})\pt\pdv{\tau_A^x}{z}
   && \Longrightarrow
   &&\mathrm{RHS}\pt\tau^x = \qty(\frac{1}{\rho_O})\pt\eval{\qty(\frac{\tau_A^x}{z})\ }_{z=0}^{z=H_1}
   && =
   &&\qty(\frac{1}{\rho_O})\qty(\frac{\Delta \tau_A^x}{\Delta z}). &&
\end{align}

Donc, on voit que l'\emph{input} d'énergie est ici dépendant de l'épaisseur de la première couche.
Généralement, le tout suit un profil logarithmique d'une épaisseur de quelques mêtres.
Donc, on ajoute effectivement plus de vitesse dans notre \(\pdv{u}{t}\), comme la couche la couche est plus mince.
Par contre, au total, le momentum général est respecté, car il prend effectivement plus d'énergie pour déplacer une plus grosse couche.\bigskip

\nb Ça serait pertinent de parler de ça à Jean-Michel au MIT. 


\section{Re-structuration des fonctions d'output -- \textit{<2023-11-14 Tue>}}
\label{sec:org9ebf897}

Dès maintenant, il est important de changer la manière qu'on lit les fichiers dans les codes Python.
Bref, je laisse cette section ici pour m'assurer que je me souvienne de ce changement.
On crée des \emph{xarray.DataArray} et on les transfert en \emph{xarray.Dataset} par la suite pour finalement créer des fichiers de type \emph{netCDF}. \bigskip

Avant, on réunissait tout avant dans un xarray dataset.


\begin{tikzpicture}
   \filldraw [orange!30] (0,0) rectangle (8,4);
   \draw (1,3.5) node [red,righ] {tls.bintods};
   \filldraw [BurntOrange!50] (0.5,0.5) rectangle (7.5,3);
   \draw (1.5,2.5) node [orange,align left] {tls.bintoda};
\end{tikzpicture}


\section{Expériences lancées -- \textit{<2023-11-15 Wed>}}
\label{sec:org2f1de08}

Comme lors de la maîtrise, il faudrait des comparatifs entre les \emph{runs} couplées et non-couplées.

\begin{center}
\begin{tabular}{clccl}
\hline
\hline
Symbole & Dénominateur & Couplage & Valeur Step & Description\\[0pt]
[ -- ] & [ -- ] & \cmark/\xmark & [ \% ] & [ -- ]\\[0pt]
\hline
A & SW-step0.0 & \xmark & 0.00 & Échantillon SW\\[0pt]
B & SW-step5.0 & \xmark & 5.00 & Échantillon SW\\[0pt]
C & COU-step0.0 & \cmark & 0.00 & Échantillon test\\[0pt]
D & COU-step5.0 & \cmark & 5.00 & Échantillon test\\[0pt]
\hline
\end{tabular}
\end{center}


\subsection{Énergie cinétique}
\label{sec:org897fa2d}


\bibliography{/home/charlesedouard/Desktop/Travail/Documentation/master-bibliography}
\end{document}