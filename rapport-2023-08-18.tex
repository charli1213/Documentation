% Created 2023-08-15 Tue 16:40
% Intended LaTeX compiler: pdflatex

% =================================BASE====================================%
\documentclass[10pt]{article}
\usepackage[left=2cm,right=2cm,top=2cm,bottom=2cm]{geometry} % Marges
%\usepackage{libertine}
%\usepackage{libertinust1math}
\usepackage[T1]{fontenc} % Nécessaire avec FrenchBabel
\usepackage[utf8]{inputenc} % Important pour symboles Francophones, é,à,etc

\usepackage{lmodern}
\renewcommand{\familydefault}{cmr} % La meilleure police (CMU Serif Roman) (Je me suis battu).
\usepackage{mathrsfs} %Permet la command \mathscr (Lettres attachées genre)

\usepackage{natbib} % Bibliographie
\bibliographystyle{abbrvnat}



\usepackage{amsmath, amssymb, amsthm} % Symb. math. (Mathmode+Textmode) + Beaux théorèmes.

\usepackage{mathtools,cancel} % Utilisation de boîtes \boxed{} + \cancelto{}{}
\usepackage{graphicx, wrapfig} % Géstion des figures.
\usepackage{hyperref} % Permettre l'utilisation d'hyperliens.
\usepackage{color} % Permettre l'utilisation des couleurs.
\usepackage[dvipsnames]{xcolor} % Couleurs avancées.
\usepackage{titling} % Donne accès à \theauthor, \thetitle, \thedate

% >>> Physique >>>
\usepackage{physics} % Meilleur package pour physicien. 
\usepackage{pxfonts} % Rajoute PLEIN de symboles mathématiques, dont les intégrales doubles et triples
% <<< Physique <<<

\usepackage{lipsum} % For fun
\usepackage{tikz} % Realisation de figures TIKZ.
\usepackage{empheq} % Boite autour de MULTIPLE équations

\usepackage[french]{babel} % Environnements en Français.
% ==============================BASE-(END)=================================%



% ================================SETTINGS=================================%
% Pas d'indentation en début de paragraphe :
\setlength\parindent{0pt} 

% Couleurs de hyperliens :
\definecolor{mypink}{RGB}{147, 0, 255}
\hypersetup{colorlinks, urlcolor=mypink, citecolor=mypink, linkcolor=mypink}

% Numéros d'équations suivent les sections :
\numberwithin{equation}{section} 

% Les « captions » sont en italique et largeur limitée
\usepackage[textfont = it]{caption} 
\captionsetup[wrapfigure]{margin=0.5cm}

% Retirer le l'écriture en gras dans la table des matières
\usepackage{tocloft}
\renewcommand{\cftsecfont}{\normalfont}
\renewcommand{\cftsecpagefont}{\normalfont}

% Change bullet style
\usepackage{pifont}
\usepackage{enumitem}
%\setlist[itemize,1]{label=\ding{224}}
\setlist[itemize,1]{label=\ding{239}}
\renewcommand{\boxtimes}{\blacksquare}
% ================================SETTINGS=================================%



% ==============================NEWCOMMANDS================================%
% Degrés Celsius :
\newcommand{\celsius}{${}^\circ$ C} % \degrée Celsius : Pas mal plus simple qu'utilise le package gensymb qui plante avec tout...

% Vecteurs de base :
\newcommand{\nvf}{\vb{\hat{n}}}
\newcommand{\ivf}{\vb{\hat{i}}}
\newcommand{\jvf}{\vb{\hat{j}}}
\newcommand{\kvf}{\vb{\hat{k}}}

\newcommand{\uu}{\vb*{u}}
\newcommand{\vv}{\vb*{v}}

% Boîte vide pour ajuster les underbrace
\newcommand{\bigno}{\vphantom{\qty(\frac{d}{q})}}
\newcommand{\pt}{\hspace{1pt}}

% Physics empty spaces 
\newcommand{\typical}{\vphantom{A}}
\newcommand{\tall}{\vphantom{A^{x^x}_p}}
\newcommand{\grande}{\vphantom{\frac{1}{xx}}}
\newcommand{\venti}{\vphantom{\sum_x^x}}

% Moyenne numérique entre deux points de grilles. 
\newcommand{\xmean}[1]{\overline{#1}^x}
\newcommand{\ymean}[1]{\overline{#1}^y}
\newcommand{\zmean}[1]{\overline{#1}^z}
\newcommand{\xymean}[1]{\overline{#1}^{xy}}

% Tilde over psi
\newcommand{\tpsi}{\tilde{\psi}}

% Nota Bene env :
\newcommand{\nb}{\ding{165}\ \textbf{N.B.}\hspace{4pt}}
   
% ==============================NEWCOMMANDS================================%



% ==============================PAGE-TITRE=================================%
% Titlepage 
\newcommand{\mytitlepage}{
\begin{titlepage}
\begin{center}
{\Large Contrat Été 2023 \par}
\vspace{2cm}
{\Large \MakeUppercase{\thetitle} \par}
\vspace{2cm}
RÉALISÉ DANS LE CADRE\\ D'UN PROJET POUR \par
\vspace{2cm}
{\Large ISMER--UQAR \par}
\vspace{2cm}
{\thedate}
\end{center}
\vfill
Rédaction \\
{\theauthor}\\
\url{charles-edouard.lizotte@uqar.ca}\\
ISMER-UQAR
\end{titlepage}
}
% ==============================PAGE-TITRE=================================%



% =================================ENTÊTE==================================%
\usepackage{fancyhdr}
\pagestyle{fancy}
\setlength{\headheight}{13pt}
\renewcommand{\headrulewidth}{1.3pt} % Ligne horizontale en haut

\fancyhead[R]{\textit{\thetitle}}
\fancyhead[L]{\ \thepage}
\fancyfoot[R]{\textit{\theauthor}}
\fancyfoot[L]{}
\fancyfoot[C]{} 
% =================================ENTÊTE==================================%
\author{Charles-Édouard Lizotte}
\date{18/08/2023}
\title{Carnet de bord, Université McGill}
\hypersetup{
 pdfauthor={Charles-Édouard Lizotte},
 pdftitle={Carnet de bord, Université McGill},
 pdfkeywords={},
 pdfsubject={},
 pdfcreator={Emacs 27.1 (Org mode 9.6.7)}, 
 pdflang={French}}
\begin{document}

\mytitlepage
\tableofcontents\newpage

\section{Reconnection au réseau de McGill -- \textit{<2023-08-14 Mon>}}
\label{sec:org7601a85}
Un petit rappel dans mes notes personnelles que mes coordonnées de McGill sont
\begin{verbatim}
>>> charles-edouard.lizotte@mcgill.ca
\end{verbatim}
et que mon \emph{mot de passe} est exactement le même que mon \emph{mot de passe} professionnel de l'UQAR.
Comme j'utilise mes adresses professionnelle et étudiante de l'UQAR, j'ai rarement besoin d'utiliser celle de McGill.
Donc, ça serait bien qu'elle soit écrite quelque part dans mon carnet de travail si je ne veux pas la perdre.


\section{Problème avec la méthode Leapfrog -- \textit{<2023-08-14 Mon>}}
\label{sec:org5e5e5cd}

\subsection{Nouveau schéma d'intégration leapfrog pour la correction de psi barotrope}
\label{sec:orgc6f5aed}
Pour clarifier la méthode \emph{leapfrog}, la procédure consiste à prendre le champ précédent (à \(t-\delta t\)) et lui additionner le double du \emph{RHS} (pris au temps \(t\)) dans le but d'obtenir le champ au temps \(t+\delta t\).
Conséquemment, la méthode se déclare comme suit, soit
\begin{equation}
   u^{t+\delta t} = u^{t-\delta t} + 2\Delta t\cdot RHS^t.
\end{equation}
Il faut impérativement trouver un moyen d'appliquer cette même méthode dans la correction de MUDPACK.
Si l'on sépare les champs, il est possible qu'on se mélange avec le \(\delta \psi\) comme il y a deux \(\Delta t\) dans la méthode \emph{leapfrog}.\bigskip

Construisons nous une suite d'étapes logiques pour appliquer la méthode \emph{leapfrog} avec MUDPACK.
\begin{enumerate}
\item On calcule \(\delta \psi\) à l'aide de la différence de \(\zeta_{BT}\) entre \(t-\delta t\) et \(t+\delta t\) (voir équation \ref{eq:org305b786}) ;
\item Trouver le \(\psi^{t+\delta t}\) à l'aide de \(\psi^{t-\delta t}\), ce qui se traduit par l'utilisation de MUDPACK pour trouver \(\delta \psi\) (voir équation \ref{eq:orgfe08445});
\item Appliquer le filtre de Robert (section \ref{orgb2d5eb0})  sur les deux quantités (\(\zeta_{BT}\) et \(\psi_{BT}\));
\item Et redéfinir \(\psi^t = \psi^{t+\delta t}\) pour le prochain time step de MUDPACK, de manière identique à ce qu'on fait dans la routine \emph{main.f90}.
\end{enumerate}

Concrétement, on se définit une méthode \emph{leapfrog} pour la correction à l'aide de MUDPACK, soit des champs à trois \emph{étages}, comme dans la fonction \emph{main.f90}.
Nous aurions 3 quantités, soient \(\psi^{t+\delta t}\), \(\psi^t\), \(\psi^{t-\delta t}\) et le \(\delta \psi\) calculé par MUDPACK serait définit par
\begin{equation}
\label{eq:orgfe08445}
   \delta \psi = \psi^{t+\delta t} - \psi^{t-\delta t},
\end{equation}
ce qui mettra la condition sur \(\zeta_{BT}\) que
\begin{equation}
\label{eq:org305b786}
   \delta \zeta_{BT} = \zeta_{BT}^{t+\delta t} - \zeta_{BT}^{t-\delta t}.
\end{equation}

\subsection{Filtre de Robert}
\label{sec:org0dde827}
\label{orgb2d5eb0}
L'application du filtre de Robert est caractérisée par le système d'équations,
\begin{align}
   &\psi^t \ \pt= \psi^t \ + R_f\pt\pt \qty( \psi^{t+\delta t}  +\psi^{t-\delta t} - 2\psi^t)\pt,\\
   &\zeta_{BT}^t = \zeta_{BT}^t + R_f\pt \qty( \zeta_{BT}^{t+\delta t}  +\zeta_{BT}^{t-\delta t} - 2\zeta_{BT}^t),
\end{align}
où \(R_f \sim 0.001\).



\section{Retour au débuggage de MUDPACK -- \textit{<2023-08-15 Tue>}}
\label{sec:orgde4b868}
Certains indices me permettent de remarquer que le problème pourrait être apparent à l'intérieur de la méthode de relaxation de MUDPACK.
En gros, on a toujours un courant nul au milieu vertical du modèle, ce qui est douteux, mais qui pourrait être un artefact de la méthode employée par MUDPACK lors de sa relaxation.
J'essaye donc d'autre schémas d'intégration, tel que la relaxation par lignes verticales, par exemple.
Mais sinon, je suis assez désespéré\ldots{}
\end{document}