% Created 2023-10-17 Tue 12:51
% Intended LaTeX compiler: pdflatex

% =================================BASE====================================%
\documentclass[10pt]{article}
\usepackage[left=2cm,right=2cm,top=2cm,bottom=2cm]{geometry} % Marges
\usepackage[T1]{fontenc} % Nécessaire avec FrenchBabel
\usepackage[utf8]{inputenc} % Important pour symboles Francophones, é,à,etc


% Calligraphie
\usepackage{lmodern}
\renewcommand{\familydefault}{cmr} % La meilleure police (CMU Serif Roman) (Je me suis battu).
\usepackage{mathrsfs} %Permet la command \mathscr (Lettres attachées genre)

% Bibliographie
\usepackage[round, sort]{natbib} % Bibliographie
\bibliographystyle{abbrvnat}


\usepackage{amsmath, amssymb, amsthm} % Symb. math. (Mathmode+Textmode) + Beaux théorèmes.
\usepackage{mathtools,cancel,xfrac} % Utilisation de boîtes \boxed{} + \cancelto{}{}
\usepackage{graphicx, wrapfig} % Géstion des figures.
\usepackage{hyperref} % Permettre l'utilisation d'hyperliens.
\usepackage{color} % Permettre l'utilisation des couleurs.
\usepackage{colortbl} % Color tables
\usepackage[dvipsnames]{xcolor} % Couleurs avancées.
\usepackage{titling} % Donne accès à \theauthor, \thetitle, \thedate

% Physique
\usepackage{physics} % Meilleur package pour physicien. 
\usepackage{pxfonts} % Rajoute PLEIN de symboles mathématiques, dont les intégrales doubles et triples

% Style
\usepackage{lipsum} % For fun
\usepackage{tikz} % Realisation de figures TIKZ.
\usepackage{empheq} % Boite autour de MULTIPLE équations

% Français
\usepackage[french]{babel} % Environnements en Français.
% ==============================BASE-(END)=================================%



% ================================SETTINGS=================================%
% Pas d'indentation en début de paragraphe :
\setlength\parindent{0pt}
\setlength{\parskip}{0.15cm}

% Tableaux/tabular
% Espace vertical dans les tabular/tableaux
\renewcommand{\arraystretch}{1.2}
% Couleur des tableaux/tabular
\rowcolors{2}{violet!5}{}

% Couleurs de hyperliens :
\definecolor{mypink}{RGB}{147, 0, 255}
\hypersetup{colorlinks, 
             filecolor=mypink,
             urlcolor=mypink, 
             citecolor=mypink, 
             linkcolor=mypink, 
             anchorcolor=mypink}



% Numéros d'équations suivent les sections :
\numberwithin{equation}{section} 

% Les « captions » sont en italique et largeur limitée
\usepackage[textfont = it]{caption} 
\captionsetup[wrapfigure]{margin=0.5cm}

% Retirer l'écriture en gras dans la table des matières
\usepackage{tocloft}
\renewcommand{\cftsecfont}{\normalfont}
\renewcommand{\cftsecpagefont}{\normalfont}

% Change bullet style
\usepackage{pifont}
\usepackage{enumitem}
%\setlist[itemize,1]{label=\ding{224}}
\setlist[itemize,1]{label=\ding{239}}
\renewcommand{\boxtimes}{\blacksquare}
% ================================SETTINGS=================================%



% ==============================NEWCOMMANDS================================%

% Vecteurs de base :
\newcommand{\nvf}{\vb{\hat{n}}}
\newcommand{\ivf}{\vb{\hat{i}}}
\newcommand{\jvf}{\vb{\hat{j}}}
\newcommand{\kvf}{\vb{\hat{k}}}
\newcommand{\uu}{\vb*{u}}
\newcommand{\vv}{\vb*{v}}

% Physics empty spaces 
\newcommand{\typical}{\vphantom{A}}
\newcommand{\tall}{\vphantom{A^{x^x}_p}}
\newcommand{\grande}{\vphantom{\frac{1}{xx}}}
\newcommand{\venti}{\vphantom{\sum_x^x}}
\newcommand{\pt}{\hspace{1pt}} % One horizontal pt space

% Moyenne numérique entre deux points de grilles. 
\newcommand{\xmean}[1]{\overline{#1}^x}
\newcommand{\ymean}[1]{\overline{#1}^y}
\newcommand{\zmean}[1]{\overline{#1}^z}
\newcommand{\xymean}[1]{\overline{#1}^{xy}}

% Tilde over psi
\newcommand{\tpsi}{\tilde{\psi}}
\newcommand{\tphi}{\tilde{\phi}}

% Nota Bene env : (\ding{89})
\newcommand{\nb}{\raisebox{0.8pt}{\scriptsize\textleaf}\ $\mathscr{N. B.}$\hspace{4pt}}
   
% ==============================NEWCOMMANDS================================%



% ==============================PAGE-TITRE=================================%
% Titlepage 
\newcommand{\mytitlepage}{
\begin{titlepage}
\begin{center}
{\Large Contrat Été 2023 \par}
\vspace{2cm}
{\Large \MakeUppercase{\thetitle} \par}
\vspace{2cm}
RÉALISÉ DANS LE CADRE\\ D'UN PROJET POUR \par
\vspace{2cm}
{\Large ISMER--UQAR \par}
\vspace{2cm}
{\thedate}
\end{center}
\vfill
Rédaction \\
{\theauthor}\\
\url{charles-edouard.lizotte@uqar.ca}\\
ISMER-UQAR
\end{titlepage}
}
% ==============================PAGE-TITRE=================================%



% =================================ENTÊTE==================================%
\usepackage{fancyhdr}
\pagestyle{fancy}
\setlength{\headheight}{13pt}
\renewcommand{\headrulewidth}{0.025pt} % Ligne horizontale en haut

\fancyhead[R]{\textit{\thetitle}}
\fancyhead[L]{\ \thepage}
\fancyfoot[R]{\textit{\theauthor}}
\fancyfoot[L]{}
\fancyfoot[C]{} 
% =================================ENTÊTE==================================%
\author{Charles-Édouard Lizotte}
\date{20/10/2023}
\title{Carnet de bord, Université McGill}
\hypersetup{
 pdfauthor={Charles-Édouard Lizotte},
 pdftitle={Carnet de bord, Université McGill},
 pdfkeywords={},
 pdfsubject={},
 pdfcreator={Emacs 27.1 (Org mode 9.6.7)}, 
 pdflang={French}}
\begin{document}

\mytitlepage
\tableofcontents\newpage
\section{Debuggage et implémentation transfert de masse -- \textit{<2023-10-16 Mon>}}
\label{sec:org09fcffb}
\subsection{Vérifier que ce n'est pas un problème de viscosité -- \textit{<2023-10-16 Mon>}}
\label{sec:org1386377}
Avant tout, David a remarqué que les champs étaient très bruités, ce qui signifie qu'il y a clairement un manque à gagner en terme de viscosité.
N'oublions pas que nous sommes passées d'une viscosité au 4ème degré vers une viscosité au second degré quand nous cherchions le problème au bord, il y a quelques semaines. \bigskip

En sommes, de nouveaux test ont été effectuées pour le schéma de viscosité exprimé par
\begin{equation}
   \vb{D} = Ah_2 \cdot \laplacian{\uu} - Ah_4\cdot \gradient^4\uu.
\end{equation}
En ce mardi matin, les résultats sont exprimés dans le tableau \ref{tab:org7b6bc5a}.



\begin{table}[htbp]
\caption{\label{tab:org7b6bc5a}Résumé des expériences réalisées dans le but de retrouver la bonne viscosité.}
\centering
\begin{tabular}{c|c|c|c|l}
\hline
Ah\textsubscript{2} & Ah\textsubscript{4} & dx & min(\sfrac{$L_d$}{dx}) & Nombre d'itér.\\[0pt]
[ -- ] & [ -- ] & [ km ] & [ -- ] & [ -- ]\\[0pt]
\hline
\hline
0.0 & (1\texttimes{}10\textsuperscript{-5})\pt\texttimes{} dx\textsuperscript{4} & 3.9 & 5.363 & 736 272 (Active)\\[0pt]
0.0 & (2\texttimes{}10\textsuperscript{-5})\pt\texttimes{} dx\textsuperscript{4} & 3.9 & 5.363 & 736 272 (Active)\\[0pt]
0.0 & (5\texttimes{}10\textsuperscript{-5})\pt\texttimes{} dx\textsuperscript{4} & 3.9 & 5.363 & 113\\[0pt]
0.0 & (1\texttimes{}10\textsuperscript{-4})\pt\texttimes{} dx\textsuperscript{4} & 3.9 & 5.363 & 48\\[0pt]
0.0 & (5\texttimes{}10\textsuperscript{-4})\pt\texttimes{} dx\textsuperscript{4} & 3.9 & 5.363 & 23\\[0pt]
\hline
\hline
\end{tabular}
\end{table}


En somme, il semble que tous nos problèmes venaient bel et bien du changement de viscosité que nous avions appliqué pour régler le problème d'ondes de Kelvin aux bord (problème qui a été réglé \href{rapport-2023-10-06.pdf}{il y a deux rapports}).\bigskip



\subsection{« Stencil » de transfert de masse}
\label{sec:orgc289194}
Louis-Philippe propose d'utiliser un stencil à 21 points pour redistribuer la masse.
En gros, on en retirerait sur le point fautif pour rejouter du \emph{h} aux points des alentours, ce qui en fait une redistribution horizontale de la masse.

\begin{figure}[!h]
\centering
\begin{tikzpicture}
   \fill [blue!5] (1,0) -- (4,0) -- (4,1) -- (5,1) -- (5,4) -- (4,4) -- (4,5) -- (1,5) -- (1,4) -- (0,4) -- (0,1) -- (1,1) -- (1,0);
   \fill [blue!12] (1,1) rectangle (4,4);
   \draw [dotted,thin] (1,0) grid (4,5);
   \draw [dotted,thin] (0,1) grid (5,4);
   \draw [] (1,0) -- (4,0) -- (4,1) -- (5,1) -- (5,4) -- (4,4) -- (4,5) -- (1,5) -- (1,4) -- (0,4) -- (0,1) -- (1,1) -- (1,0);
   \fill [cyan!50] (2,2) rectangle (3,3); 
   \draw [] (2,2) rectangle (3,3);
   %
   \draw (2.5,2.5) node {+1};
\end{tikzpicture}
\caption{\label{orgab9ca32}Stencil de redistribution de la masse.}
\end{figure}

\section{Solution à la dérive de Stokes -- \textit{<2023-10-16 Mon>}}
\label{sec:orgd725159}
Grossièrement, il est sorti deux possibilités pour régler le problème des petites échelles qui sortent de Wavewatch :
\begin{itemize}
\item Il serait possible de diminuer la résolution de Wavewatch et de réinterpoler les points de courants à l'aide de la méthose employée dans la figure \ref{org3d68a6c}.

\begin{figure}[h!]
\begin{center}
\begin{tikzpicture}
% Big grid
\fill [blue!5] (0,0) rectangle (3,3);
\fill [blue!5] (3,3) rectangle (6,6);
% Grid
\draw (0,0) rectangle (6,6) ;
\draw [dotted] (0,0) grid (6,6) ;
\draw [step=3.0] (0,0) grid (6,6) ;
% Carré
\draw [cyan, thick] (2,2) rectangle (5,5) ;
\fill [cyan!50, opacity=0.5] (3,3) rectangle (4,4);
% Coordinates 
\foreach \x in {1,2,3}
\foreach \y in {1,2,3}
{\draw (\x-0.5,\y-0.5) node [] {1,1};}
%
\foreach \x in {4,5,6}
\foreach \y in {1,2,3}
{\draw (\x-0.5,\y-0.5) node [] {2,1};}
%
\foreach \x in {1,2,3}
\foreach \y in {4,5,6}
{\draw (\x-0.5,\y-0.5) node [] {1,2};}
%
\foreach \x in {4,5,6}
\foreach \y in {4,5,6}
{\draw (\x-0.5,\y-0.5) node [] {2,2};}
% Axis:
\foreach \y in {1,2,3,4,5,6} {\draw (-0.5,\y-0.5) node [cyan] {\y};}
\foreach \x in {1,2,3,4,5,6} {\draw (\x-0.5,-0.5) node [cyan] {\x};}
%
\end{tikzpicture}
\end{center}
\begin{equation}
   {\color{cyan}u(4,4)} = \qty(\sfrac{1}{9})\cdot u(1,1) + \qty(\sfrac{2}{9})\cdot u(2,1) + \qty(\sfrac{2}{9})\cdot u(1,2) + \qty(\sfrac{4}{9})\cdot u(2,2).
\end{equation}
\caption{\label{org3d68a6c}« Stencil » utilisé pour obtenir le champs aux plus grandes échelles.}
\end{figure}
\end{itemize}




\bibliography{/home/charlesedouard/Desktop/Travail/Documentation/master-bibliography}
\end{document}