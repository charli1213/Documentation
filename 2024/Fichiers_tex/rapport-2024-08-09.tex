% Created 2024-09-03 Tue 12:35
% Intended LaTeX compiler: pdflatex
\documentclass[10pt]{article}
% =================================BASE====================================%
\usepackage[left=2cm,right=2cm,top=2cm,bottom=2cm]{geometry} % Marges
\usepackage[T1]{fontenc} % Nécessaire avec FrenchBabel
\usepackage[utf8]{inputenc} % Important pour symboles Francophones, é,à,etc
\usepackage{csquotes} % Recommandé par PDFLatex lors de la compilation. 

% Calligraphie
%\usepackage{pxfonts} % Met le texte ET les maths en Palatino + donne accès à des symboles math
%\usepackage{palatino} % Cette commande met seulement le texte en police palatino
\usepackage{lmodern} % Pour les maths? Lmodern pour les maths
\usepackage{cfr-lm}
% Use lmodern for sans-serif
\usepackage{mathrsfs} % Permet la command \mathscr (Lettres attachées genre) \mathscr(ABC)
\usepackage{eucal}   % Vient changer le \mathcal{ABC} parce que celui de base est laid.


% Bibliographie
\usepackage[backend=biber,style=apa,sorting=ynt]{biblatex}
%\usepackage[backend=biber,sorting=ynt,style=authoryear]{biblatex} % Ça semble tout changer.
\addbibresource{master-bibliography.bib}


\usepackage{amsmath, amssymb, amsthm} % Symb. math. (Mathmode+Textmode) + Beaux théorèmes.
\usepackage{mathtools,cancel,xfrac} % Utilisation de boîtes \boxed{} + \cancelto{}{}, xfrac
\usepackage{graphicx, wrapfig} % Géstion des figures.
\usepackage{hyperref} % Permettre l'utilisation d'hyperliens.
\usepackage{color} % Permettre l'utilisation des couleurs.
\usepackage{colortbl} % Color tables
\usepackage[dvipsnames]{xcolor} % Couleurs avancées.

% Physique
\usepackage{physics} % Meilleur package pour physicien. 

% Style
\usepackage{lipsum} % For fun
\usepackage{tikz} % Realisation de figures TIKZ.
\usetikzlibrary{arrows.meta,bending} % Arrow heads 
\usepackage{empheq} % Boite autour de MULTIPLE équations
\usepackage{bbding}

% Français
\usepackage[french]{babel} % Environnements en Français.

\usepackage{titling} % Donne accès à \theauthor, \thetitle, \thedate

% ==============================BASE-(END)=================================%





% ================================SETTINGS=================================%
% Pas d'indentation en début de paragraphe :
\setlength\parindent{0pt}
\setlength{\parskip}{0.15cm}

% Tableaux/tabular
% Espace vertical dans les tabular/tableaux
\renewcommand{\arraystretch}{1.2}
% Couleur des tableaux/tabular
% \rowcolors{3}{violet!5}{}

% Couleurs de hyperliens :
\definecolor{mypink}{RGB}{147, 0, 255}
\hypersetup{colorlinks, 
             filecolor=mypink,
             urlcolor=mypink, 
             citecolor=mypink, 
             linkcolor=mypink, 
             anchorcolor=mypink}


% Numéros d'équations suivent les sections :
\numberwithin{equation}{section} 

% Les « captions » sont en italique et largeur limitée
\usepackage[textfont = it]{caption} 
\captionsetup[wrapfigure]{margin=0.5cm}

% Retirer l'écriture en gras dans la table des matières
\usepackage{tocloft}
\renewcommand{\cftsecfont}{\normalfont}
\renewcommand{\cftsecpagefont}{\normalfont}

% Change bullet style
\usepackage{pifont}
\usepackage{enumitem}
%\setlist[itemize,1]{label=\ding{224}}
\setlist[itemize,1]{label=\ding{239}}
\renewcommand{\boxtimes}{\blacksquare}
% ================================SETTINGS=================================%



% ==============================NEWCOMMANDS================================%
% CQFD symbol
\renewcommand{\qedsymbol}{$\hfill\blacksquare$}

% Vecteurs de base :
\newcommand{\nvf}{\vb{\hat{n}}}
\newcommand{\evf}{\vb{\hat{e}}}
\newcommand{\ivf}{\vb{\hat{i}}}
\newcommand{\jvf}{\vb{\hat{j}}}
\newcommand{\kvf}{\vb{\hat{k}}}
\newcommand{\uu}{\vb{u}}
\newcommand{\vv}{\vb{v}}
\newcommand{\ust}{\vb{u}_{\ast}}

% Physics empty spaces 
\newcommand{\short}{\vphantom{pA}}
\newcommand{\tall}{\vphantom{pA^{x^x}_p}}
\newcommand{\grande}{\vphantom{\frac{1}{xx}}}
\newcommand{\venti}{\vphantom{\sum_x^x}}
\newcommand{\pt}{\hspace{1pt}} % One horizontal pt space

% Moyenne numérique entre deux points de grilles. 
\newcommand{\xmean}[1]{\overline{#1}^x}
\newcommand{\ymean}[1]{\overline{#1}^y}
\newcommand{\zmean}[1]{\overline{#1}^z}
\newcommand{\xymean}[1]{\overline{#1}^{xy}}

% Tilde over psi
\newcommand{\tpsi}{\tilde{\psi}}
\newcommand{\tphi}{\tilde{\phi}}

% Nota Bene env : (\ding{89})
%\newcommand{\nb}{$\boxed{\text{\footnotesize\EightStarConvex}\pt \mathfrak{N. B.}}$\hspace{4pt}}
\newcommand{\nb}{\underline{{\footnotesize\EightStarConvex}\pt $\mathfrak{N.B.}$\vphantom{p}}\hspace{3pt}}

\newcommand{\exemple}{
\parbox[center]{2.2cm}{\begin{tcolorbox}[sharp corners, rounded corners=northeast, rounded corners=southeast,
colback=Violet!2, colframe=black,
size=small, width=2cm, left=-0.25pt, bottom=-0.5pt,
arc is angular, arc=2.5mm, boxrule=0.35pt, leftrule=4pt, %bottomrule=1pt,
after={\enskip}] Exemple \end{tcolorbox}}}

\newcommand{\rad}{\text{Rad}}


\newcommand{\cqfd}{\hfill$\blacktriangleleft$}

% Define the nota bene environment
\usepackage{tcolorbox}
\newtcolorbox{notabene}{
     colback=blue!5,
     colframe=black,
     boxrule=0.5pt,
     arc=2pt,
     left=5pt,
     right=5pt,
     top=5pt,
     bottom=5pt,
}


\newcommand{\cmark}{\ding{52}}
\newcommand{\xmark}{\ding{55}}
% ==============================NEWCOMMANDS================================%



% ==============================PAGE-TITRE=================================%
% Titlepage 
\newcommand{\mytitlepage}{
\begin{titlepage}
\begin{center}
{\Huge \thesubtitle \par}
\vspace{2cm}
{\Huge \MakeUppercase{\thetitle} \par}
\vspace{2cm}
RÉALISÉ DANS LE CADRE\\ D'UN PROJET POUR \par
\vspace{2cm}
{\Huge ISMER--UQAR \par}
\vspace{2cm}
{\thedate}
\end{center}
\vfill
Rédaction \\
{\theauthor}\\
\url{charles-edouard.lizotte@uqar.ca}\\
ISMER-UQAR\\
Police d'écriture : \textbf{CMU Serif Roman}
\end{titlepage}
}
% ==============================PAGE-TITRE=================================%



% =================================ENTÊTE==================================%
\usepackage{fancyhdr}
\pagestyle{fancy}
\setlength{\headheight}{13pt}
\renewcommand{\headrulewidth}{0.025pt} % Ligne horizontale en haut

\fancyhead[R]{\textit{\thetitle}}
\fancyhead[L]{\quad\thepage}
\fancyfoot[R]{\textit{\theauthor}}
\fancyfoot[L]{}
\fancyfoot[C]{} 
% =================================ENTÊTE==================================%
\author{Charles-Édouard Lizotte}
\date{09/08/2024}
\title{Rapport hebdomadaire}
\newcommand{\thesubtitle}{Contrat Été 2024}
\hypersetup{
 pdfauthor={Charles-Édouard Lizotte},
 pdftitle={Rapport hebdomadaire},
 pdfkeywords={},
 pdfsubject={},
 pdfcreator={Emacs 29.4 (Org mode 9.7.8)}, 
 pdflang={French}}
\begin{document}

\mytitlepage
\tableofcontents\newpage
\section{Notes sur la programmation du modèle WIM en Julia}
\label{sec:org23d8782}

\subsection{Opérateur linéaire de Lax-Wendroff}
\label{sec:org9ca8214}
La méthode de Lax-Wendroff peut aussi être exprimée selon le formalisme d'un opérateur linéaire, de telle sorte que
\begin{equation}
   \underset{nx\times nx}{\mathbf{A}}\cdot\underset{nx\times nf}{\vb{E}} = \begin{pmatrix}
       a_{11} & a_{12} & \cdots & a_{1,nx} \\
       a_{21} & a_{22} & \cdots & a_{2,nx} \\
       \vdots & \ddots &  & \vdots \\
       a_{nx,1} & a_{nx,2} & \cdots & a_{nx,nx} \\
   \end{pmatrix}\begin{pmatrix}
       E_{1,1} & E_{1,2} & \cdots & E_{1,nf}\\ 
       E_{2,1} & E_{1,2} & \cdots & E_{2,nf}\\
       \vdots & \ddots &  & \vdots\\
       E_{nx,1} & E_{1,2} & \cdots & E_{2,nf}\\
   \end{pmatrix}
\end{equation}
L'advection avec Lax-Wendroff est définit comme
\begin{equation}
   E_i^{n+1} = E_i^{n} - \qty(\frac{\Delta t}{2 \Delta x}) c_g (E_{i+1}^n - E_{i-1}^n) + \qty(\frac{\Delta t^2}{2 \Delta x^2}) c_g^2 (E_{i+1}^n -2E_i^n + E_{i-1}^n).
\end{equation}
On isole les variables,
\begin{align}
   \Delta E_i(\omega) = \underbrace{\venti-\lambda^2 c_g(\omega)}_{\boxed{A}} E_i^n + \underbrace{\venti\frac{\lambda c_g(\omega)}{2} \qty(1 + \lambda c_g(\omega))}_{\boxed{B}}E_{i-1} + \underbrace{\venti\frac{\lambda c_g(\omega)}{2} \qty(\lambda c_g(\omega) -1)}_{\boxed{C}} E_{i+1}.
\end{align}
Puis finalement,
\begin{equation}
   \Delta \mathbf{E}(\omega) = \mathbf{A}(\omega)\cdot\vb{E} = \begin{pmatrix}
       A & C & 0 & \cdots & 0 \\
       B & A & C & \cdots & 0 \\
       \vdots & & & \ddots & \vdots\\
       0 & \cdots & 0 & B & A \\
     \end{pmatrix}\begin{pmatrix}
       E(\omega,x_1) \\
       E(\omega,x_2) \\
       \vdots\\
       E(\omega,x_{nf}) \\
     \end{pmatrix},
\end{equation}
et comme la matrice \textbf{A} est différente pour chaque fréquence, il faut donc recalculer \textbf{A} pour chaque fréquence.
Notons que dans la maîtrise de Bismuth, on assumait une vitesse de groupe identique pour tous les fréquences, ce que je trouve très ordinaire, personellement.
\subsection{Lax-Wendroff aux murs}
\label{sec:orgf7fc3df}
À la case initiale, on assume une condition frontière, donc pas besoin de Lax-Wendroff là.
Puis à la case finale, on assume une frontière ouverte de sorte que
\begin{equation}
   \eval{\qty(\pdv{E}{x} = 0)}_{x = x_f + \Delta x/2}
\end{equation}
\section{Rappel sur les propriétés des vagues}
\label{sec:org21f37da}

\subsection{Équations principales}
\label{sec:org2e7fb59}

La fréquence absolue (\(\omega = 2\pi f_a\)) est reliée à la fréquence relative (\(\sigma = 2\pi f_r\)) à l'aide de la relation \autocite{wwiii2016user,Ardhuin2024ocean},
\begin{equation}
   \boxed{\grande\quad\omega = \sigma + \vb{k}\cdot\vb{u}.\quad}
\end{equation}
Tiré du \emph{poster} de Dany, la densité d'action des vagues \(N(\omega, \theta; \vb{x}, t)\) (\emph{Wave action density spectrum}) est reliée au spectre d'énergie \(E(\omega, \theta; \vb{x}, t, x)\) (\emph{Energy density spectrum}) à l'aide de la relation
\begin{equation}
   \boxed{\ \grande E = \omega N.\ }
\end{equation}
Grossièrement, on garde généralement la densité d'action des vagues parce que c'est une quantité conservée, si je me souviens bien. 
\subsection{La vitesse d'une onde}
\label{sec:org5caa36b}
Quelques rappels techniques rapides, 
\begin{align}
   \omega = 2\pi \cdot f && k = 2\pi / \lambda && \omega = \sigma + \mathbf{k}\cdot \uu 
\end{align}

La relation de dispersion des ondes de surfaces est aussi exprimée par
\begin{equation}
\label{eq:orgde57527}
   \sigma = \qty(gk\cdot \tanh(kd))^{\sfrac{1}{2}}
\end{equation}

Selon Wiki, la vitesse de phase est donnée par
\begin{equation}
   v_p = \frac{\omega}{\mathbb{R} \qty{k}},
\end{equation}
donc
\begin{equation}
   \boxed{\quad c_p(\omega) = \sqrt{\frac{g}{k}} = \frac{g}{\omega}\quad}.
\end{equation}
\subsection{Lien entre la vitesse de groupe et la vitesse de phase}
\label{sec:org167c655}

Toujours selon Wiki, on définit la vitesse de groupe à l'aide de
\begin{equation}
   v_g = \dv{\omega}{k}
\end{equation}
Donc, à l'aide de la relation de dispersion et sachant que \(\uu = 0\), on obtient
\begin{align}
   \dv{\omega}{k} &= \sqrt{g} \cdot \frac{1}{2} k^{-\sfrac{1}{2}},\nonumber\\
     &= \frac{1}{2}\sqrt{\frac{g}{k}},\nonumber\\
\end{align}
ainsi on peut mettre en relation la vitesse de groupe et de phase, soit
\begin{equation}
   \boxed{\quad c_p = \venti\frac{g}{\omega}\hspace{0.5cm};\hspace{0.5cm} c_g = \frac{c_p}{2}.\quad}
\end{equation}
Mentionnons aussi que le vecteur \(\mathbf{c}_g(\omega)\) est fixe dans le temps, il n'y a pas de dispersion temporelle.
\subsection{Désambiguation des quantités utilisée dans le domaine d'étude des vagues}
\label{sec:org33b2bf0}

\subsubsection{Petit retour historique}
\label{sec:org4a6db0b}

À la base -- ce qu'on a -- c'est la hauteur du niveau de l'eau (que nous appellerons \(h(\vb{x},t)\)).
Il existe un niveau de l'eau moyen établi sur tout le domaine, mais ce n'est pas important, car c'est l'évolution des écarts à la moyenne qui nous intéresse.
Donc, 
\begin{equation}
   h(\vb{x},t) = \overline{h}(t) + h'(\vb{x},t) \enskip\mid\enskip \overline{h}(t) = \expval{h(\vb{x},t)}
\end{equation}
Quand on parles des écarts à la moyenne, c'est l'écart-type ou la \textbf{variance} qui nous vient en tête. 
La \textbf{variance} est donnée par la moyenne des carrés des écarts à la moyenne, soit
\begin{align}
   \mathrm{Variance} &= \expval{\qty(h - \expval{h})^2},\nonumber\\
   &= \expval{\qty(h^2 -2h\expval{h} + \expval{h}^2)}\\
   &= \expval{h^2 + \expval{h}^2} -2 \expval{h}\expval{h}\\
   &= \expval{h^2} - \expval{h}^2.
\end{align}

D'ici, on peut faire la transformée de Fourier du spectre de variance \(h^2(\vb{x},t)\) dans une case, de sorte à obtenir \(h^2(\vb{k}, \omega; \vb{x},t)\).
L'énergie d'une onde est proportionnelle au carré de l'amplitude de celle-ci, donc à l'aide du théorème de Parceval, on peut relier l'amplitude des coefficient de notre série de Fourier à l'énergie associée à chaque fréquence.
Dans une \textbf{transformée de Fourier}, on a 
\begin{equation}
   E(\vb{x},t; \vb{k},\omega) \propto h^2(\vb{x},t; \vb{k}, \omega).
\end{equation}
C'est pourquoi, on utilise souvent l'expression \emph{variance density spectrum} pour parler du spectre d'énergie.
Bref, ces deux quantités sont similaires.
\subsubsection{Retour sur les quantités}
\label{sec:org1a303b5}
Wavewatch utilise Pour citer la documentation des Wavewatch \autocite{wwiii2016user} :\medskip
\begin{quote}
\emph{For monochromatic waves, the amplitude is described as the amplitude, the wave height, or the wave energy. For irregular wind waves, the (random) variance of the sea surface is described using variance density spectra (in the wave modeling community usually denoted as energy spectra). The variance spectrum F is a function of all independent phase parameters, i.e.,} \(F(\vb{k},\sigma,\omega)\), \emph{and furthermore varies in space and time at scales larger than those of individual waves, e.g.,} \(F(\vb{k},\sigma,\omega; x, t)\). [\ldots{}]
\emph{Within WAVEWATCH III the basic spectrum is the wavenumber-direction spectrum} \(F(k,\theta)\), \emph{which has been selected because of its invariance characteristics with respect to physics of wave growth and decay for variable water depths. The output of WAVEWATCH III, however, consists of the more traditional frequency-direction spectrum} \(F(f_r,\theta)\).
\end{quote}
\medskip

et puis, on mentionne aussi \medskip
\begin{quote}
\emph{In a general sense, however, wave action} \(A\equiv E/\sigma\) \emph{is conserved (e.g., Whitham, 1965; Bretherthon and Garrett, 1968). This makes the wave action density spectrum} \(N(k,\theta) = F(k,\theta)/\sigma\) \emph{the spectrum of choice within the model.}
\end{quote}
\medskip

Sans courant, la variance ou l'énergie n'est pas conservé, c'est pourquoi on utilise le spectre de densité d'action, au final.


\begin{center}
\begin{tabular}{cllc}
\hline
Symbole & Quantité & Anglais & Relation\\
\hline
\(E\) & Spectre d'énergie des vagues & \emph{Energy spectrum} & ---\\
\(F\) & Spectre de variance des vagues & \emph{Variance spectrum} & ---\\
\(N\) & Spectre de densité d'action des vagues & \emph{Wave Action density spectrum} & \(N = F/\sigma\)\\
\(A\) & Densité d'action & \emph{Action} & \(A = E/\sigma\)\\
\hline
\end{tabular}
\end{center}
\section{Débriefing réunion Dany (16 aout 2024)}
\label{sec:org5140feb}

\subsection{Théorie linéaire des vagues dans la glace?}
\label{sec:org8e7544a}
Actuellement, il n'existe pas vraiment de théorie linéaire des vagues ou de formalisme des vagues à l'intérieur de la glace comme dans \Textcite{miles1957generation}.
Mentionnons qu'il y a un article de \Textcite{miles1996surface} décrivant les vagues dans un milieu visco-élastique, mais je n'ai pas encore eu le temps de le lire.\bigskip

Peut-être qu'il y a un moyen simple de faire ça, mais considérant les mathématiques assez avancées de \Textcite{miles1957generation}, j'en doute.
Faudrait peut-être revisiter de manière plus claire ces trucs-là.
\subsection{Modélisation des vagues en 1 dimension}
\label{sec:org0a0dfe4}
Très difficile de représenter les interactions  entre les vagues dans un «set-up» en 1 dimension : 
\begin{itemize}
\item Le \emph{spreading} dans les fréquences est mal définit.
Quand on fait références aux interactions vagues-vagues, c'est à ça qu'on fait référence (triplète et quadruplètes, aussi).
Il devrait statistiquement y avoir un transfert d'énergie des hautes vers les basses fréquences.
\item Dany mentionnait que dans le modèle Wavewatch III, c'était précisément la \emph{switch} NL4 que Sébastien Dugas avait réussi dans une version de WIM subséquente à intégrer les interactions en 1 dimension, mais les termes de son expansion polynomiale étaient très obscurs.
Selon Dany, c'est basé sur les travaux de Hasselmann, mais je pense qu'on parle plus de NL2, dans ce cas.
\item Le \emph{spreading} dans les directions est inexistant en 1 dimension.
Par contre, Dany avait amené l'argument aux reviewers que l'on peut voir une front d'onde comme la propagation de multiples cercles, précisément comme dans l'explication des fentes de Young.
\end{itemize}
\subsection{Planification}
\label{sec:org8cc02ab}
Grossièrement, comment ferions nous pour définir un coefficient d'hétérogénéité associé à l a glace.
Au premier ordre, la glace n'est qu'un filtre et on a tout intérêt à la considérer comme telle.
S'il y a de la glace, on filtre des ondes.
Déjà, on peut comparer l'effet de mettre de la glace à haute résolution (50\% de glace par exemple, avec des concentrations de glace de 1) et de voir si l'on peut comparer ça avec un modèle à basse résolution (Concentration de glace de 0.5). \bigskip

Le motif principal est qu'on ne connait pas dutout les paramètrisations qui ont été faites avec les autres termes sources pour la glace, si c'étaient des \emph{patch} de glace ou de grandes étendues, par exemple.
Dany a amené un argument intéressant :\smallskip
\begin{quote}
\emph{C'est jamais du tout ou rien, s'il y a de la glace, ça ne veut pas dire que les vagues peuvent ne pas croître dans une certaine mesure. Justement, dans la maîtrise d'Eliot Bismuth, c'est ce qu'on a tenté de faire. La relation entre la concentration et le couvert de glace ne devrait pas du tout être linéaire.}
\end{quote}
Donc, on va trouver un moyen de définir un coefficient (ou une matrice) qui prend en compte :
\begin{itemize}
\item La distribution spatiale de nos glaces;
\item La concentration et l'épaisseur de glace.
\end{itemize}
\end{document}
